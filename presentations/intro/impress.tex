\documentclass{beamer}

\setbeamertemplate{navigation symbols}{}
\usetheme{Montpellier}
\beamersetuncovermixins{\opaqueness<1>{25}}{\opaqueness<2->{15}}

\begin{document}
\title{Bayesian parameter synthesis of  Markov population models.}  
\author{Nhat-Huy Phung}
\date{\today} 

\begin{frame}
  \titlepage
\end{frame}

\begin{frame}
  \frametitle{Table of contents}
  \tableofcontents
\end{frame}

\section{Motivation}
\begin{frame}
  \frametitle{Case study}
  We studies a bees colony
  \begin{itemize}
  \item Bees response to stimulations from the environment by \textit{stinging}.
  \item After \textit{stinging}, an individual bee releases \textit{pheromone}
    and dies.
  \end{itemize}
  As we study the biological system, we have the following research questions:
  \begin{enumerate}
  \item Given a population of bee, how many individuals left in the steady
    state.
  \item How does an individual's behaviour affect the population size at the
    steady state? If each individual bee is more aggresive, would it increase or
    decrease the population size at steady states?
  \end{enumerate}
  In the scope of this thesis we use bees colony as a case study. However, the
  method can be applied to general Discrete-Time Markov Chain models.
\end{frame}

\begin{frame}
  \frametitle{Data}
  \begin{itemize}
  \item \textbf{Biological data} is collected from a real bee colony, which is
    currently nurtured by Department of Biology.
  \item \textbf{Synthetic data} is generated by simulating the parametric DTMC
    using a concrete assignment of parameters.
  \end{itemize}
  Using \textit{synthetic data} has an advantage over using real data. Namely,
  as the concrete parameters are known, it is possible to measure the distance
  between the synthesized parameters and true parameters.
\end{frame}

\section{Model}
\begin{frame}
  \frametitle{Approach}
  \begin{itemize}
  \item Model individual's behaviour as a Markov Decision Process.
  \item Model collective behaviour as a composition of many individual
    behavioural models.
  \item Answer the research question by checking the collective behaviour model
    against properties represented in the form of temporal logic.
  \end{itemize}
\end{frame}

\begin{frame}
  \frametitle{Single agent model.}
  We model the behaviour of a single bee as a Markov Decision Process. Let
  $\mathcal{S}$ be the individual model, we have
  \begin{align*}
    \mathcal{S} = (S, A, P_a, R_a)
  \end{align*}
  in which
  \begin{itemize}
  \item $S$ is the set of states.
  \item $A$ is the set of actions.
  \item $P_a(s,s')$ is the probability of transitioning from state $s$ to state
    $s'$ given action $a$.
  \item $R_a(s,s')$ is the \textit{reward} received after transitioning from
    state $s$ to state $s'$ given action $a$.
  \end{itemize}
\end{frame}

\begin{frame}
  \frametitle{Collective model}
  To model collective behavior of multiple agents, we construct
  product of individual models. Let $\mathcal{M}$ be the multiple agents model,
  we have
  \begin{align*}
    \mathcal{M} = (\mathcal{S}_1||\mathcal{S}_2||\ldots||\mathcal{S}_k) 
  \end{align*}
  in which $k$ is the population size and $||$ denotes synchronous composition.\\
  As Markov Decision Process can be seen as a probabilistic automata,
  construction of MDPs composition follows cite[Sokolova].
\end{frame}

\section{Properties}
\begin{frame}
  \frametitle{Population properties}
  \textbf{Question:} How large is the population size at the steady state?\\
  \textbf{Answer:} In our model, as each absorbing state represents a population
  size at the steady state, the question can be answered by checking the model
  against PCTL properties.
  \begin{align*}
    P_{?}(FG s_i)
  \end{align*}
\end{frame}


\begin{frame}
  \frametitle{Proposed Framework}
  We propose a framework to synthesize model parameters. based on 
  \begin{itemize}
  \item 
  \end{itemize}
\end{frame}

