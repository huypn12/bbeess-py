\chapter{Bayesian inference}
We present essential concepts in Bayesian parameter inference and several methods to estimate
posterior distribution. The methods range from posterior conjugations, in which tractability is
guaranteed as we know the analytic form of both likelihood and prior distribution. Afterwards, we
discuss different sampling algorithm to approximate the posterior distribution when no conjugations
are available. We also present a likelihood-free method to exploit in the case that the analytical
form of the likelihood is not achievable or is too complex to evaluate. The sampling algorithms
presented in this chapter are the building block for the Bayesian frameworks that we present in this
thesis.

\section{Bayesian inference}
\subsection{Bayesian formula}
Let $D_{obs}$ be observed data. In statistical inference, we assume that the observed data has a
probability distribution of unknown parameter $\theta$, that is $D_{obs} \sim P(D_{obs}|\theta)$. There are two main approaches in statistical inference
\begin{enumerate}
    \item Frequentist approach.
    \item Bayesian approach.
\end{enumerate}
In frequentist approach, the estimation of $\theta$ based on long-run property, that is, given a
large enough sample size, expected value of parameter estimation $\hat{\theta}$ is equal to
$\theta$. Therefore, frequentist approach requires to gather a large amount of data to deliver a
close estimation $\hat{\theta}$.  The main advantage of Bayesian approach over frequentist approach
is that it require less data to obtain an estimation $\hat{\theta}$.\\
In Bayesian approach, we use the information gained from previously observed data \textit{(beliefs)}
to enhance the accuracy of the estimation of $\hat{\theta}$. The beliefs obtained from prior
knowledge of model parameter $\theta$ is represented by \textit{prior distribution} $\pi(\theta)$.
We have the \textit{likelihood} $P(D_{obs}|\theta)$ as the probability distribution over observed
data, given parameter $\theta$.
\begin{definition}{Bayes theorem}
    \begin{align*}
        \pi(\theta | D_{obs}) = \frac{P(D_{obs}|\theta)\pi(\theta)}{\int_\theta P(D_{obs}|\theta)\pi(\theta)d\theta}
    \end{align*}
    where
    \begin{itemize}
        \item $\int_\theta P(D_{obs}|\theta)\pi(\theta)d\theta$ is the \textit{marginal distribution}.
        \item $\pi(\theta | D_{obs})$ is the \textit{posterior distribution}
    \end{itemize}
\end{definition}
The essential part of Bayesian inference in statistic is to compute or estimate the posterior
distribution. From the analytical form or the samples from the posterior distribution, we estimate
the model parameter $\theta$.

\subsection{Bayesian parameter estimation}
With posterior distribution $\pi(\theta|D)$ we estimate the parameter $\hat{\theta}$ using Bayesian
posterior mean.
\begin{definition}{Bayesian posterior mean}
    \begin{align*}
        \hat{\theta} = \mathbf{E}[\theta] = \int_\theta \theta \pi(\theta|D) d\theta
    \end{align*}
\end{definition}
In case we have samples from posterior distribution, for example a trace $T$ of parameter values
${\theta_1,\dots,\theta_|T|}$ from Metropolis-Hastings algorithm, the discrete form of posterior
mean is used:
\begin{align*}
    \hat{\theta} = \mathbf{E}[\theta] = \sum_\theta \theta \pi(\theta|D)
\end{align*}

\begin{definition}[Bayesian Credible Set]
    Set C is a $(1 − \alpha )100\%$ credible set for the parameter $\theta$ if the posterior
    probability for $\theta$ to belong to C equals $(1 − \alpha)$.
    \begin{align*}
        P(\theta \in C | D) = \int_C \pi(\theta|D) d\theta = 1 - \alpha
    \end{align*}
\end{definition}
In this thesis, we use by default $0.95$ credible set, which corresponds to $\alpha=0.05$
\begin{definition}[Highest Posterior Density credible set]
    Highest Posterior Density $(1-\alpha)100\%$ credible set (HPD for short) is the
    interval with minimum length over all Bayesian $(1-\alpha)100\%$ Credible Set.
\end{definition}

In this research, the HPD is calculated using algorithm from \textit{PyMC3} library
\cite{salvatier2016pymc3}. For simplicity, we assume that in all cases which we concern, HPD is
computed using the algorithm for unimodal distribution.
\begin{algorithm}[H]
    \caption{Compute Highest Posterior Density Interval}
    \label{mh}
    \hspace*{\algorithmicindent} \textbf{Input:} $S$ is samples from a distribution. \\
    \hspace*{\algorithmicindent} \textbf{Input:} $0\leq \alpha \leq 1$ \\
    \hspace*{\algorithmicindent} \textbf{Output:} HPD interval
    \begin{algorithmic}[1]
        \Procedure{Compute HPD}{$S$}
        \State Compute interval width $w = |S| * \alpha$
        \State Find modal (peak) of sample points.
        \State Return minimal interval of size $|S| - w$ which contains the modal.
        \EndProcedure
    \end{algorithmic}
\end{algorithm}

\subsection{Selection of prior distribution}
Theoretically, prior can be of any distribution family. However, a selection of prior distribution
that is too different than the actual distribution of parameter can leads to a false propagation of
beliefs and degrade inference results.\\
It is suggested by \cite{polgreen2016data} that in case of no prior knowledge exists to help the
selection of prior distribution, Uniform distribution is preferable since it is less likely to
propagate false beliefs to the inference.\\
A systematic inference to select prior distribution family and prior distribution parameter
(hyperparameters) is possible with \textit{Hierarchical Bayes Models}
\cite{allenby2005hierarchical}.

\subsection{Estimation of posterior distribution}
\subsubsection{Posterior conjugation}
Conjugated posteriors are special cases of Bayesian inference, in which the prior and posterior
distribution belongs to the same family of distribution. When posterior conjugation is applicable,
only the parameters of probability distribution function need to be re-estimated. Applying
conjugated posterior when it is possible gives advantages:
\begin{itemize}
    \item Tractability: we have analytical form of posterior distribution with only changes in its
          parameters.
    \item Computationally effective: updating model parameter is of linear time to the dimension of
          parameter.
\end{itemize}
We consider two conjugated posteriors as examples: Binomial-Beta and Dirichlet-Multinomial.
\begin{lemma}[Binomial-Beta Conjugation]
    Binomial distribution is conjugated to beta distribution.
\end{lemma}
\begin{proof}
    The observed data $D=(x_1,\ldots,x_n)$ is sampled from $Binomial(k, \theta)$ function
    \begin{align*}
        P(D|\theta) = \prod_{i=1}^n{k\choose x_i}\theta^{x_i}(1-\theta)^{k-x_i}
    \end{align*}
    The parameter $\theta$ is of $Beta(\alpha, \beta)$ distribution
    \begin{align*}
        \pi(\theta) = \theta^{\alpha-1}(1-\theta)^{\beta -1}
    \end{align*}
    We obtained:
    \begin{align*}
        \pi(\theta|D) & \sim P(D|\theta)\pi(\theta)                                                                             \\
                      & \sim \theta^{\sum_{i=1}^n x_i}(1-\theta)^{nk -\sum_{i=1}^n x_i} \theta^{\alpha -1} (1-\theta)^{\beta-1} \\
                      & = \theta^{\alpha - 1 + \sum_{i=1}^n x_i}(1-\theta)^{\beta - 1 + nk -\sum_{i=1}^n x_i}
    \end{align*}
    Thus, the posterior is $Beta(\alpha + \sum_{i=1}^n x_i, \beta + nk -\sum_{i=1}^n x_i)$
\end{proof}
Generalize this conjugation, we also have Multinomial-Dirichlet conjugation.
\begin{lemma}[Multinomial-Dirichlet Conjugation]
    Multinomial distribution is conjugated to Dirichlet distribution.
\end{lemma}
\begin{proof}
    The observed data $D=(x_1,\ldots,x_n)$ is sampled from $Multinomial(n; \theta_1,\ldots,\theta_n)$ function
    \begin{align*}
        P(x_1,\ldots,x_n | N, \theta_0,\ldots,\theta_n) & = \frac{n!}{x_1!\ldots x_n!} \prod_{i=1}^n\theta_i^{x_i}
    \end{align*}
    The parameter $(\theta_1,\ldots,\theta_n)$ is
    $Dirichlet(\alpha_1,\ldots,\alpha_n)$
    \begin{align*}
        \pi(\theta_1,\ldots,\theta_n) = \frac{1}{\mathbf{B}(\alpha_1,\ldots,\alpha_n)}\prod_{i=1}^n\theta_i^{\alpha_i - 1}
    \end{align*}
    We obtain
    \begin{align*}
        \pi(\theta_1,\ldots,\theta_n|D) & \sim P(D|\theta)\pi(\theta)                                           \\
                                        & \sim \prod_{i=1}^n\theta_i^{x_i} \prod_{i=1}^n\theta_i^{\alpha_i - 1} \\
                                        & \sim \prod_{i=1}^n\theta_i^{\alpha_i - 1 + \sum_{i=1}^n x_i}
    \end{align*}
    Thus, the posterior is $Dirichlet(\alpha_1 +  x_1,\ldots,\alpha_n
        +  x_n)$
\end{proof}
More detailed description in these cases can be found in \cite{tu2014dirichlet}
and \cite{baron2019probability}. We summarize the necessary results in the following table:
\begin{table}[H]
    \begin{tabular}{lllll}
        \cline{1-3}
        \multicolumn{1}{|l|}{Likelihood}                                 & \multicolumn{1}{l|}{Prior}                                 & \multicolumn{1}{l|}{Posterior parameters}                         &  & \\ \cline{1-3}
        \multicolumn{1}{|l|}{$Binomial(n, k)$}                           & \multicolumn{1}{l|}{$Beta(\alpha, \beta)$}                 & \multicolumn{1}{l|}{\begin{tabular}[x]{@{}c@{}}$\alpha' = \alpha + \sum_{i=1}^n x_i$\\$\beta' = \beta + nk -\sum_{i=1}^n x_i$\end{tabular}}                   &  & \\ \cline{1-3}
        \multicolumn{1}{|l|}{$Multinomial(n; \theta_1,\ldots,\theta_n)$} & \multicolumn{1}{l|}{$Dirichlet(\alpha_1,\ldots,\alpha_n)$} & \multicolumn{1}{l|}{$\alpha_i' =\alpha_i + x_i, 1 \leq i \leq n$} &  & \\ \cline{1-3}
                                                                         &                                                            &                                                                   &  &
    \end{tabular}
\end{table}
However, posterior conjugation is applicable to a subset of prior and likelihood functions. In
Bayesian inference, it is usual that the posterior distribution has no analytical form or its
analytical form is difficult to directly sample from. In these cases, we can several different
sampling and optimization methods to approximate the posterior distribution. In the following
section we discuss different approaches for posterior distribution approximation:
\begin{itemize}
    \item Markov chain Monte-Carlo.
    \item Sequential Monte-Carlo.
    \item Approximate Bayesian Computation.
\end{itemize}

\subsubsection{Markov chain Monte-Carlo}
In case the posterior distribution has no analytical form or its analytical form is difficult to
sample from directly, we use \textit{Metropolis-Hastings} algorithm  (\textit{MH} in short).\\
invented by Metropolis \cite{metropolis1953equation} and later generalized by Hastings \cite{hastings1970monte}


Metropolis-Hastings algorithm is a \textit{Monte Carlo Markov Chain} algorithm. In its essential,
Metropolis-Hastings algorithm draws sample from an unknown distribution. Using the MH algorithm, we
can estimate the parameter by posterior mean, without knowing the analytical form of posterior
distribution itself.

\begin{algorithm}[H]
    \caption{Metropolis-Hastings Algorithm}
    \label{alg:mh}
    \hspace*{\algorithmicindent} \textbf{Input:}
    \begin{itemize}
        \item Model $\mathcal{M}_\theta$
        \item $D_{obs}$: observation data
        \item Likelihood function $P(D|\theta)$
        \item $\pi(\theta)$: prior distribution
        \item Transition kernel $Q(\theta^t|\theta^{t-1})$
        \item $N$ number of particles.
    \end{itemize}
    \hspace*{\algorithmicindent} \textbf{Output:}
    \begin{itemize}
        \item $(\theta_1,\ldots,\theta_N)$ sample of $N$ particles
        \item $(w_1,\ldots,w_N)$ corresponding likelihoods.
    \end{itemize}
    \begin{algorithmic}[1]
        \Procedure{Metropolis-Hastings}{$D$, maxIteration}
        \State Draw $\theta_0$ from $\pi(\theta)$
        \State $i \leftarrow 0$
        \While{maxIteration not reached}
        \State $L \leftarrow P(D|\theta)$
        \State Draw a point $\theta' $ from the proposal distribution.
        \State $L' \leftarrow P(D|\theta')$
        \If{ $\ln(L') - \ln(L) > 0$ }
        \State Add $\theta'$ to $Trace$
        \State $\theta = \theta'$
        \Else
        \State Draw a random number $x$ from $Uniform(0,1)$
        \If{$x \leq \xi$, ($\xi$ very small, e.g $10^{-8}$)}
        \State Add $\theta'$ to $Trace$ (avoiding local maxima)
        \State $\theta = \theta'$
        \EndIf
        \EndIf
        \EndWhile
        \EndProcedure
    \end{algorithmic}
\end{algorithm}

The likelihood function can be implemented as log-likelihood to avoid underflow error.
Advantages of Metropolis-Hastings are
\begin{itemize}
    \item Parameter transition only needs the computation of likelihood function.
          Therefore, Monte Carlo Markov Chain can be used in general Bayesian inference,
          in which we are not guaranteed to have an analytical form of posterior.
    \item Computationally efficient; as marginal distribution is cancelled out, and likelihood can
          be replaced by log-likelihood, Metropolis-Hastings simplifies the computation of Bayes formula
          and avoid infinitesimal values.
    \item Simple to implement.
\end{itemize}
Disadvantages of Metropolis-Hastings are
\begin{enumerate}
    \item Particle in Metropolis-Hastings algorithm moves in a linear Markov chain; it is highly
          probable to be stuck in a local maximum or minimum.
    \item Not parallelizable; since there is only one linear chain, and current step depends on
          previous step, Metropolis-Hastings algorithm does not scale up to multi-processors.
\end{enumerate}
The next algorithm, \textit{Sequential Monte-Carlo}, address the issues of Metropolis-Hastings.

\subsubsection{Sequential Monte-Carlo}
Sequential Monte-Carlo method is firstly proposed by \cite{del2006sequential}.  Instead of having
one particle moving in its parameter space, Sequential Monte-Carlo estimates by using $N$ particles
moving independently. Sequential Monte-Carlo uses  starting by a \textit{pertubation kernel}.

Therefore Sequential Monte-Carlo method has a significant advantage of easily
parallelizable. Comparision between \cite{daviet2018inference}
good for multimodal
\begin{algorithm}[H]
    \caption{Sequential Monte-Carlo Algorithm}
    \label{alg:smc}
    \hspace*{\algorithmicindent} \textbf{Input:}
    \begin{itemize}
        \item Model $\mathcal{M}_\theta$
        \item $D_{obs}$: observation data
        \item $\pi(\theta)$: prior distribution
        \item $P(D|\theta)$: Likelihood function.
        \item Transition kernel $Q(\theta^t|\theta^{t-1})$
        \item \item Transition kernel $Q(\theta^t|\theta^{t-1})$
        \item $N$ number of particles.
    \end{itemize}
    \hspace*{\algorithmicindent} \textbf{Output:}
    \begin{itemize}
        \item $(\theta_1,\ldots,\theta_N)$ sample of $N$ particles
        \item $(w_1,\ldots,w_N)$ corresponding likelihoods.
    \end{itemize}
    \begin{algorithmic}[1]
        \Procedure{Metropolis-Hastings}{$D$, maxIteration}
        \State Select a proposal distribution $\pi(\theta)$
        \State Draw a random initial point $\theta$
        \State Init empty trace $Trace$
        \While{maxIteration not reached}
        \State $L \leftarrow P(D|\theta)$
        \State Draw a point $\theta' $ from the proposal distribution.
        \State $L' \leftarrow P(D|\theta')$
        \If{ $\ln(L') - \ln(L) > 0$ }
        \State Add $\theta'$ to $Trace$
        \State $\theta = \theta'$
        \Else
        \State Draw a random number $x$ from $Uniform(0,1)$
        \If{$x \leq \xi$, ($\xi$ very small, e.g $10^{-8}$)}
        \State Add $\theta'$ to $Trace$ (avoiding local maxima)
        \State $\theta = \theta'$
        \EndIf
        \EndIf
        \EndWhile
        \EndProcedure
    \end{algorithmic}
\end{algorithm}
Selection of kernel function for SMC is mentioned in \cite{silk2012optimizing}.

\subsubsection{Approximate Bayesian Computation}
The methods mentioned before is used with an assumption that the likelihood $P(D_{obs}|\theta)$ has
an analytical form; the analytical can be evaluated without introducing computational burden.
However for situations in which the likelihood has no analytical form, or the analytical form is
expensive to be evaluated, we use a class of \textit{likelihood-free} methods. Likelihood-free
methods in Bayesian inference estimates the likelihood $P(D_{obs}|\theta)$ ,
estimate it or replace it by other measures. Approximate Bayesian Computation is a widely used
likelihood-free method for approximating posterior distribution. Instead of estimating the
likelihood $P(D|\theta)$ directly, we sample a observable data set $\hat{D}$ and define a distance
measure $\delta(D, \hat{D})$. Approximate Bayesian Computation accepts a set of tuples
$(\hat{\theta}, \hat{D})$, each satisfies that $\delta(D_{obs},D_{sim}) < \epsilon,
    \epsilon\in\mathbf{R}_{\leq 0}$.
\begin{algorithm}[H]
    \caption{Approximate Bayesian Computation}
    \label{alg:abc-reject}
    \hspace*{\algorithmicindent} \textbf{Input:}
    \begin{itemize}
        \item Model $\mathcal{M}_\theta$
        \item $D_{obs}$: observation data
        \item $\pi(\theta)$: prior distribution
        \item $P(D|\theta)$: Likelihood function.
        \item Transition kernel $Q(\theta^t|\theta^{t-1})$
        \item \item Transition kernel $Q(\theta^t|\theta^{t-1})$
        \item $N$ number of particles.
    \end{itemize}
    \hspace*{\algorithmicindent} \textbf{Output:}
    \begin{itemize}
        \item $(\theta_1,\ldots,\theta_N)$: $N$ sampled particles.
        \item $(w_1,\ldots,w_N)$: corresponding weights of sampled particles.
    \end{itemize}
    \begin{algorithmic}[1]
        \Procedure{Approximate-Bayesian-Computation}{}
        \State Select a proposal distribution $\pi(\theta)$
        \State Draw a random initial point $\theta$
        \State Init empty trace $Trace$
        \While{maxIteration not reached}
        \State $L \leftarrow P(D|\theta)$
        \State Draw a point $\theta' $ from the proposal distribution.
        \State $L' \leftarrow P(D|\theta')$
        \If{ $\ln(L') - \ln(L) > 0$ }
        \State Add $\theta'$ to $Trace$
        \State $\theta = \theta'$
        \Else
        \State Draw a random number $x$ from $Uniform(0,1)$
        \If{$x \leq \xi$, ($\xi$ very small, e.g $10^{-8}$)}
        \State Add $\theta'$ to $Trace$ (avoiding local maxima)
        \State $\theta = \theta'$
        \EndIf
        \EndIf
        \EndWhile
        \EndProcedure
    \end{algorithmic}
\end{algorithm}

\section{Summary}
We present a set of optimization and approximation methods which are essentials to Bayesian
Inference. In the following chapter we propose a data-driven approach for parameter synthesis
combining Approximate Bayesian computation, Sequential Monte Carlo, and Statistical Model Checking.
