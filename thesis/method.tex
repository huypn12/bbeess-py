\chapter{Bayesian frameworks for parameter synthesis.}
We present frameworks for data-informed parameter synthesis of pDTMC. The frameworks are designed to
synthesize a set of parameter values so that for each value, the instantiated model satisfies the
interested property. Formally, given a pDTMC model $\mathcal{M}_\theta$, a PCTL property $\Phi$, and
observed data $D_{obs}$, the frameworks synthesize a set of $N$ parameters
$(\theta_1,\ldots,\theta_N)$ such that
\begin{align*}
      \forall i \in [1,N]: \mathcal{M}_{\theta_i} \models \Phi
\end{align*}
There are two frameworks designed towards two different use cases:
\begin{enumerate}
      \item Symbolic rational functions are available for both BSCC reachability properties and $\Phi$.
      \item Only simulation and statistical model checking are possible.
\end{enumerate}

\section{Generic framework}
We present the generic frameworks for Bayesian parameter synthesis of pDTMC. The generic takes a
pDTMC, a property of interest and observed data as input. In the core of the framework, we use
Sequential Monte Carlo to  The generic framework based on Sequential Monte Carlo algorithm. However,
particular implementations of Sequential Monte Carlo
\begin{algorithm}[H]
      \caption{Generic framework for Bayesian parameter synthesis}
      \label{alg:generic-framework}
      \footnotesize{
            \hspace*{\algorithmicindent} \textbf{Input:}
            \begin{itemize}
                  \item $\mathcal{M}_\theta$: parametric Discrete-Time Markov chain of parameter $\theta$
                  \item $\Phi$: bounded reachability property of interest.
                  \item $D_{obs}$: observed data.
                  \item $N$: number of particles.
            \end{itemize}
            \hspace*{\algorithmicindent} \textbf{Output:}
            \begin{itemize}
                  \item $(\theta_1,\ldots,\theta_{N_{MH}})$: $N_{MH}$ sampled particles.
                  \item $(w_1,\ldots,w_{N_{MH}})$: corresponding weights of sampled particles.
            \end{itemize}
      }
      \begin{algorithmic}[1]
            \Procedure{Generic-Bayesian-Monte Carlo}{}
            \State $i \leftarrow 1$
            \While{$i \leq N$}
            \State Sample $\theta$ by Sequential Monte Carlo sampling algorithm.
            \State Verify instantiated model $\mathcal{M}_\theta$ against $\Phi$
            \If{$\mathcal{M}_\theta \models \Phi$}
            \State $\theta_i \leftarrow \theta$
            \EndIf
            \EndWhile
            \State Return $(\theta_1,\ldots,\theta_N)$
            \EndProcedure
      \end{algorithmic}
\end{algorithm}

\subsection{Selection of pertubation kernel}
An analysis on the seletion of pertubation kernel is presented by Silk \cite{silk2012optimizing}. For
simplicity and avoiding the propagation of false beliefs, we select uniform, component-wise
pertubation kernel. In the Metropolis-Hastings step when the analytical form of likelihood is
available, we implement Single Component Metropolis-Hastings. The transition kernel
$Q(\theta^t|\theta^{t-1})$ is selected component-wise, identical to the pertubation kernel, however
the minimum and maximum values are extracted from local Metropolis-Hasting trace.

\section{Symbolic computation based frameworks}
As we have analytical form for both target property and likelihood function, the framework is
designed identical to the original Sequential Monte Carlo algorithm (Del \cite{del2006sequential}). The
only difference is that our framework only accept parameter values that instantiate satisfying
concrete DTMC models. As in Sequential Monte Carlo sampler, the first step is to define
Metropolis-Hastings step for each independent particle $\theta$.

\begin{algorithm}[H]
      \caption{Metropolis-Hastings with rational functions}
      \label{alg:rf-mcmc}
      \footnotesize{
            \hspace*{\algorithmicindent} \textbf{Input:}
            \begin{itemize}
                  \item $\mathcal{M}_\theta$: parametric Discrete-Time Markov chain of parameter $\theta$
                  \item $\Phi$: bounded reachability property of interest.
                  \item $\pi(\theta)$: prior distribution on $\theta$.
                  \item $N_{MH}$: length of particle trace.
                  \item $Q(\theta^t|\theta^{t-1})$: transition kernel.
                  \item $D_{obs}$: observed data.
                  \item $P(D_{obs}|\theta):$ likelihood function.
            \end{itemize}
            \hspace*{\algorithmicindent} \textbf{Output:}
            \begin{itemize}
                  \item $(\theta_1,\ldots,\theta_{N_{MH}})$: $N_{MH}$ sampled particles.
                  \item $(w_1,\ldots,w_{N_{MH}})$: corresponding weights of sampled particles.
            \end{itemize}
      }
      \begin{algorithmic}[1]
            \Procedure{RF-MCMC}{}
            \State $sat \leftarrow False$
            \While{$sat = False$}
            \State Draw $\theta_{cand}$ from $\pi(\theta)$
            \State Evaluate $val \leftarrow RF_{\Phi}(\theta)$
            \If{$val$ satisfies the boundary of $\Phi$}
            \State $sat \leftarrow True$
            \EndIf
            \EndWhile
            \State $\theta_1 \leftarrow  \theta_{cand}$
            \State $w_1 \leftarrow  \ln(P(D_{obs}|\theta_{cand}))$
            \State $i \leftarrow 2$
            \While{$i \leq N_{MH}$}
            \State $sat \leftarrow False $
            \While{$sat = False$}
            \State Draw $\theta_{cand}$ from $Q(\theta'|\theta_{i-1})$
            \State Evaluate $val \leftarrow RF_{\Phi}(\theta)$
            \If{$val$ satisfies the boundary of $\Phi$}
            \State $sat \leftarrow True$
            \EndIf
            \EndWhile
            \If{ $\ln(P(D_{obs}|\theta_{cand})) - \ln(P(D_{obs}|\theta_{i-1})) > 0$ }
            \State $\theta_i \leftarrow \theta_{cand}$
            \State $w_i \leftarrow \ln(P(D_{obs}|\theta_{cand}))$
            \State $i \leftarrow i + 1$
            \Else
            \State Draw a random number $u$ from $Uniform(0,1)$
            \If{$u \leq \xi$, ($\xi$ small, e.g $10^{-2}$)}
            \State $\theta_i \leftarrow \theta_{cand}$
            \State $w_i \leftarrow \ln(P(D_{obs}|\theta_{cand}))$
            \State $i \leftarrow i + 1$
            \EndIf
            \EndIf
            \EndWhile
            \State Return $(\theta_1,\ldots,\theta_{N_{MH}})$, $(w_1,\ldots,w_{N_{MH}})$
            \EndProcedure
      \end{algorithmic}
\end{algorithm}

\begin{algorithm}[H]
      \caption{Sequential Monte Carlo with rational functions}
      \label{alg:rf-smc}
      \footnotesize{
            \hspace*{\algorithmicindent} \textbf{Input:}
            \begin{itemize}
                  \item $\mathcal{M}_\theta$: parametric Discrete-Time Markov chain of parameter $\theta$
                  \item
                  \item $\Phi$: bounded reachability property of interest.
                  \item $\pi(\theta)$: prior distribution on $\theta$.
                  \item $N$: number of particles in the Sequential Monte Carlo trace.
                  \item $M$ pertubation kernels $F_t(\theta^t | \theta^{t-1}_1,\ldots,\theta^{t-1}_N), 1\leq t \leq M$
                  \item $N_{MH}$: number of particles in each Metropolis-Hastings step.
                  \item $Q_t(\theta^t|\theta^{t-1}), 1 \leq t \leq N_{MH}$: transition kernel for Metropolis-Hastings step.

            \end{itemize}
            \hspace*{\algorithmicindent} \textbf{Output:}
            \begin{itemize}
                  \item $(\theta_1,\ldots,\theta_N)$: $N$ sampled particles.
                  \item $(w_1,\ldots,w_N)$: corresponding weights of sampled particles.
            \end{itemize}
      }
      \begin{algorithmic}[1]
            \Procedure{RF-SMC}{}
            \State $i \leftarrow 1$
            \While{$i \leq N$} \algorithmiccomment {SMC initialization}
            \State Draw $\theta$ from $\pi(\theta)$
            \State $\theta_i \leftarrow \theta$
            \State $w_i \leftarrow P(D_{obs}|\theta_i)$
            \State $i \leftarrow i + 1$
            \EndWhile
            \State $t \leftarrow 1$
            \While{$t \leq M$}
            \State $i \leftarrow 1$ \algorithmiccomment{SMC correction step}
            \While{$i \leq N$}
            \State $w'_i \leftarrow \frac{w_i}{\sum_{i=1}^N w_i} $
            \EndWhile
            \State Sample with replacement $(\theta'_1,\ldots,\theta'_N)$ \algorithmiccomment{SMC selection step} \\\hspace{1.5cm} from $(\theta_1,\ldots,\theta_N)$ with probabilities $(w'_1,\ldots,w'_N)$
            \State $(\theta_1,\ldots,\theta_N) \leftarrow (\theta'_1,\ldots,\theta'_N)$
            \State $i \leftarrow 1$
            \While{$i \leq N$} \algorithmiccomment {SMC pertubation step}
            \State Draw $\hat{\theta}^t_i$ from $F_t(\theta^t | \theta^{t-1}_1,\ldots,\theta^{t-1}_N), 1\leq t \leq M$
            \State $(\theta^*_1,\ldots,\theta^*_{N_{MH}}), (w^*_1,\ldots,w^*_{N_{MH}}) \leftarrow RF-MCMC(\hat{\theta}^t_i)$
            \State $\theta_i \leftarrow \theta^*_{N_{MH}}$
            \State $w_i \leftarrow w^*_{N_{MH}}$
            \EndWhile
            \EndWhile
            \State Return $(\theta_1,\ldots,\theta_{N})$, $(w_1,\ldots,w_{N})$
            \EndProcedure
      \end{algorithmic}
\end{algorithm}
Since rational functions for the interested property $\Phi$ is available, as well as the analytical
form of the likelihood $P(D_{obs}|\theta)$, we design a framework based Sequential Monte Carlo
sampling algorithm with Metropolis-Hastings algorithm on pertubation step.

\section{Simulation based frameworks.}
Without the availability of analytical form of observational and interested properties, we face the
following obstacles:
\begin{itemize}
      \item \textbf{Absence of likelihood functions:} As the rational functions for properties are
            not available, we do not have the analytical form of likelihood. The abscence of
            likelihood suggests to exploit \textit{likelihood-free methods}. In this framework we
            use \textit{Approximate Bayesian Computation} in combination with \textit{Sequential
                  Monte Carlo method}.
      \item \textbf{Absence of rational function for verification of bounded reachability property:}
            the satisfaction of an instantiated model to a bounded path property cannot be computed.
            In the case that the number of states is too large, we use \textit{Statistical Model
                  Checking}.
\end{itemize}
For this case we present Statistical Model Checking, Approximate Bayesian Computation - Sequential
Monte Carlo method \textit{SMC-ABC-SMC} framework. SMC-ABC-SMC differs from RF-SMC only on
pertubation step.
\begin{itemize}
      \item In SMC-ABC-SMC framework, we work with a \textit{likelihood-free} setup, in which
            there is no analytical form of the likelihood. As there is no likelihood function, we apply
            Approximate Bayesian Computation and  accept the first particle whose simulation satisfies the
            distance threshold.
      \item There is also no rational function for the property
            of interest $\Phi$, so we apply Statistical model checking with confidence $\alpha$ and indifference
            width $\delta$.
\end{itemize}

\begin{algorithm}[H]
      \caption{Sequential Monte Carlo with Approximate Bayesian Computation and Statiscal Model Checking}
      \label{smc-abc-smc-alg}
      \footnotesize{
            \hspace*{\algorithmicindent} \textbf{Input:}
            \begin{itemize}
                  \item $\mathcal{M}_\theta$: pDTMC of parameter $\theta$
                  \item $\Phi$: bounded reachability property of interest.
                  \item $D_{obs}$: observed data
                  \item $\pi(\theta)$: prior distribution on $\theta$.
                  \item $N$: number of particles in the Sequential Monte Carlo trace.
                  \item $M$ pertubation kernels $F_t(\theta^t | \theta^{t-1}_1,\ldots,\theta^{t-1}_N), 1\leq t \leq M$
                  \item $\epsilon$: distance threshold for Approximate Bayesian Computation.
                  \item $\delta, \alpha$: indifference and $\alpha$-level for Statistical Model Checking using SPRT method.
            \end{itemize}
            \hspace*{\algorithmicindent} \textbf{Output:}
            \begin{itemize}
                  \item $(\theta_1,\ldots,\theta_N), (w_1,\ldots,w_N)$: $N$ sampled particles and corresponding weights.
            \end{itemize}
      }
      \begin{algorithmic}[1]
            \Procedure{SMC-ABC-SMC}{}
            \State $i \leftarrow 1$
            \While{$i \leq N$} \algorithmiccomment {SMC initialization}
            \State Draw $\theta$ from $\pi(\theta)$
            \State $\theta_i \leftarrow \theta$, $w_i \leftarrow 1$
            \EndWhile
            \State $t \leftarrow 1$
            \While{$t \leq M$}
            \State $i \leftarrow 1$ \algorithmiccomment{SMC correction step}
            \While{$i \leq N$}
            \State $w'_i \leftarrow \frac{w_i}{\sum_{i=1}^N w_i} $
            \EndWhile
            \State Sample (with replacement) $(\theta^t_1,\ldots,\theta^t_N)$ \\
            \hspace{3cm} from $(\theta^{t-1}_1,\ldots,\theta^{t-1}_N)$, $(w^{t-1}_1,\ldots,w^{t-1}_N)$  \algorithmiccomment{SMC selection step}
            \State $i \leftarrow 1$
            \While{$i \leq N$} \algorithmiccomment {SMC pertubation step}
            \State $rejected \leftarrow True$
            \While{$rejected == True$}
            \State $sat \leftarrow False $
            \While{$sat = False$}
            \State Draw $\hat{\theta}^t_i$ from $F_t(\theta^t | \theta^{t-1}_1,\ldots,\theta^{t-1}_N), 1\leq t \leq M$
            \If{$SPRT-SMC(\mathcal{M}_{\hat{\theta}^t}, \Phi, \epsilon, \delta)$ is SAT}
            \State $sat \leftarrow True$
            \EndIf
            \EndWhile
            \State Simulate $D_{sim}$ from $(\mathcal{M}_{\hat{\theta}^t})$
            \If{$Distance(D_{sim}, D_{obs}) < \epsilon$}
            \State $rejected \leftarrow False$
            \State $\theta_i \leftarrow \hat{\theta}^t$, $w_i \leftarrow d$
            \EndIf
            \EndWhile
            \EndWhile
            \EndWhile
            \State Return $(\theta_1,\ldots,\theta_{N})$, $(w_1,\ldots,w_{N})$
            \EndProcedure
      \end{algorithmic}
\end{algorithm}

\section{Summary}
In this chapter we presents two Sequential Monte Carlo based frameworks, which based on rational
functions and simulations. In the following chapter, we benchmark the frameworks using different
parametric DTMC to evaluate their performances.