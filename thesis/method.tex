\chapter{Bayesian frameworks for parameter synthesis.}
We present frameworks for data-informed parameter synthesis of parametric DTMC. The frameworks are
designed to synthesize a set of parameter values so that for each value, the instantiated model
satisfies the property of interest. Formally, given a parametric DTMC model $\mathcal{M}_\theta$, a
PCTL property $\Phi$, and steady-state data $D_{obs}$ collected from experiment, the frameworks
synthesize a set of $N$ parameter values $(\theta_1,\ldots,\theta_N)$ such that
\begin{align*}
    \forall i \in \{1,\ldots,N\}: \mathcal{M}_{\theta_i} \models \Phi
\end{align*}
The set of satisfying parameter values $(\theta_1,\ldots,\theta_N)$ has corresponding weights
$(w_1,\ldots,w_N)$. The weight $w_i, 1\leq i\leq N$ represents the likelihood
$P(D_{obs}|\theta_i)$ or its approximation.\\
We design two frameworks for two different use cases. In the first use cases, symbolic rational
functions are available for (i) the evaluation of steady-state distribution, and (ii) the evaluation
of property $\Phi$. In the second use case, we assume that there is no rational function obtainable
for steady-state distribution and $\Phi$. Instead, we (i) use simulation runs to estimate the
steady-state distribution, and (ii) use statistical model checking to verify the property of
interest $\Phi$.

\newpage
\section{Generic framework}
We present the generic framework for Bayesian parameter synthesis of parametric discrete-time Markov
chain (concept shown in Fig. ~\ref{fig:generic-framework}). The generic framework takes a
parametric discrete-time Markov chain, a property of interest, and steady-state data as input. In
the core of the framework, we use Sequential Monte Carlo to sample the parameter space. The generic
framework is based on the Sequential Monte Carlo algorithm. However, in the specific implementations
of the generic framework, there are two important differences in each iteration of the Sequential
Monte Carlo algorithm.
\begin{enumerate}
    \item \textit{Computing likelihood}: do we compute the (i) exact likelihood, or (ii) do we
          approximate it?
    \item \textit{Model checking the property of interest}: do we use (i) rational functions to
          evaluate the property of interest $\Phi$, or (ii) do we use statistical model checking?
\end{enumerate}
\begin{algorithm}[H]
    \caption{Generic framework for Bayesian parameter synthesis}
    \label{alg:generic-framework}
    \footnotesize{
        \hspace*{\algorithmicindent} \textbf{Input:}
        \begin{itemize}[noitemsep,topsep=0pt]
            \item $\mathcal{M}_\theta$: parametric Discrete-Time Markov chain of parameter $\theta$
            \item $\Phi$: bounded reachability property of interest.
            \item $D_{obs}$: observed data.
            \item $N$: number of particles.
        \end{itemize}
        \hspace*{\algorithmicindent} \textbf{Output:}
        \begin{itemize}[noitemsep,topsep=0pt]
            \item $(\theta_1,\ldots,\theta_{N_{MH}})$, $(w_1,\ldots,w_{N_{MH}})$: $N_{MH}$ sampled
                  particles and their corresponding weights.
        \end{itemize}
    }
    \begin{algorithmic}[1]
        \Procedure{Generic-Bayesian-Monte Carlo}{}
        \State $i \leftarrow 1$
        \While{$i \leq N$}
        \State Sample $\theta$ with corresponding weight $w$
        \NoNumber{\hspace{1.5cm} by Sequential Monte Carlo sampling algorithm.}
        \State Verify instantiated model $\mathcal{M}_\theta$ against $\Phi$
        \If{$\mathcal{M}_\theta \models \Phi$}
        \State $\theta_i \leftarrow \theta$
        \State Estimate $w_i$ as exact or approximated likelihood $P(D_{obs}|\theta)$
        \EndIf
        \EndWhile
        \State Estimate $\hat{\theta}$ using posterior mean.
        \State Compute $\hat{p}=P(\mathcal{M}_{\hat{\theta}}\models\Phi)$
        \State Return $(\theta_1,\ldots,\theta_{N})$, $(w_1,\ldots,w_{N})$, $\hat{\theta}$, $\hat{p}$
        \EndProcedure
    \end{algorithmic}
\end{algorithm}

\begin{figure}[H]
    \centering
    \begin{tikzpicture}
        \tikzset{
            node distance=1cm,
            auto,
            every text node part/.style={align=center}
        }
        \tikzstyle{block}=[
        font=\footnotesize,
        draw=black,
        rounded corners,
        fill=white,
        inner sep=0.3cm,
        outer sep=0cm,
        thick,
        ]
        \node[block] (smc) {Sample using\\Sequential Monte Carlo\\$\theta$};
        \node[block, above=of smc]  (model) {Parametric DTMC\\$\mathcal{M}_\theta$};
        \node[block, left=of model] (prop) {Property of interest\\$\Phi$};
        \node[block, right=of model] (data) {Steady-state data\\$D_{obs}$};
        \node[block, below right=of smc] (llh) {Exact or approximate\\ likelihood\\$P(D_{obs}|\theta)$};
        \node[block, below left=of smc] (mc) {Model checking\\instantiated model\\$\mathcal{M}_\theta\models\Phi$};
        \node[block, below=of llh] (result) {Parameter trace \\$(\theta_1,\ldots,\theta_N)$\\Parameter weights \\$(w_1,\ldots,w_N)$\\Parameter estimation $\hat{\theta}$};
        \node[block, left=of result] (validate) {Validate estimated parameter\\$\mathcal{M}_{\hat{\theta}}\models\Phi$ ?};

        \draw (prop.south) |- (90:15mm) -|  (smc.north);
        \draw[-latex] (model.south) --  (smc.north);
        \draw (data.south) |- (90:15mm)  -| (smc.north);

        \draw[-latex] (llh.north) |- (smc.east);
        \draw[-latex] (smc.west) |-(180:30mm)-|  (mc.north);
        \draw[-latex] (mc.east) -- (llh.west);
        \draw[-latex] (llh.south) -- (result.north);
        \draw[-latex] (result.west) -- (validate.east);
    \end{tikzpicture}
    \caption{Generic framework for Bayesian parameter synthesis of parametric DTMC.}
    \label{fig:generic-framework}
\end{figure}

\subsection{Selection of perturbation kernel}
An analysis on the seletion of perturbation kernel is presented by Filippi
\cite{filippi2013optimality} and Silk \cite{silk2012optimizing}. As there is no prior knowledge on
the covariance among multi-dimensional parameters, we select component-wise perturbation kernel, in
which the perturbation function applies to each dimension independently. We also use uniform
perturbation kernel to avoid the propagation of false beliefs. Assume model parameter is of
$k$-dimensional, $\theta=(\theta_1,\ldots,\theta_k)$. Each component $\theta_i$, $i\in [1,k]$ is
perturbated at pertubation $t$ using an uniform kernel $Uniform(\theta_i-\delta, \theta_i+\delta)$, where
\begin{align*}
    \delta = \frac{1}{2}(\max(\theta^1_i,\ldots,\theta^{t-1}_i) - \min(\theta^1_i,\ldots,\theta^{t-1}_i))
\end{align*}
In the SMC algorithm with exact likelihood is available, we select Metropolis-Hastings transition
kernel $Q(\theta^t|\theta^{t-1})$ identical to the perturbation kernel.

\section{RF-SMC framework}
We design a Sequential Monte Carlo framework with rational functions, using a modified
Metropolis-Hastings algorithm to mutate particles in each perturbation.

\begin{algorithm}[H]
    \caption{Metropolis-Hastings with rational functions}
    \label{alg:rf-mcmc}
    \footnotesize{
        \hspace*{\algorithmicindent} \textbf{Input:}
        \begin{itemize}[noitemsep,topsep=0pt]
            \item $\mathcal{M}_\theta$: parametric Discrete-Time Markov chain of parameter $\theta$
            \item $\Phi$: bounded reachability property of interest.
            \item $\pi(\theta)$: prior distribution on $\theta$.
            \item $D_{obs}$: observed data.
            \item $P(D_{obs}|\theta):$ likelihood function.
            \item $N_{MH}$: length of particle trace.
            \item $Q(\theta^t|\theta^{t-1})$: transition kernel.
        \end{itemize}
        \hspace*{\algorithmicindent} \textbf{Output:}
        \begin{itemize}[noitemsep,topsep=0pt]
            \item $(\theta_1,\ldots,\theta_{N_{MH}})$, $(w_1,\ldots,w_{N_{MH}})$: $N_{MH}$ sampled
                  particles and their weights.
        \end{itemize}
    }
    \begin{algorithmic}[1]
        \Procedure{RF-MH}{}
        \State $sat \leftarrow False$
        \While{$sat = False$}
        \State Draw $\theta_{cand}$ from $\pi(\theta)$
        \State Evaluate $val \leftarrow RF_{\Phi}(\theta)$
        \If{$val$ satisfies the boundary of $\Phi$}
        \State $sat \leftarrow True$
        \EndIf
        \EndWhile
        \State $\theta_1 \leftarrow  \theta_{cand}$
        \State $w_1 \leftarrow  \ln(P(D_{obs}|\theta_{cand}))$
        \State $i \leftarrow 2$
        \While{$i \leq N_{MH}$}
        \State $sat \leftarrow False $
        \While{$sat = False$}
        \State Draw $\theta_{cand}$ from $Q(\theta'|\theta_{i-1})$
        \State Evaluate $val \leftarrow RF_{\Phi}(\theta)$
        \If{$val$ satisfies the boundary of $\Phi$}
        \State $sat \leftarrow True$
        \EndIf
        \EndWhile
        \If{ $\ln(P(D_{obs}|\theta_{cand})) - \ln(P(D_{obs}|\theta_{i-1})) > 0$ }
        \State $\theta_i \leftarrow \theta_{cand}$
        \State $w_i \leftarrow \ln(P(D_{obs}|\theta_{cand}))$
        \State $i \leftarrow i + 1$
        \Else
        \State Draw a random number $u$ from $Uniform(0,1)$
        \If{$u \leq \xi$, ($\xi$ small, e.g $10^{-2}$)}
        \State $\theta_i \leftarrow \theta_{cand}$
        \State $w_i \leftarrow \ln(P(D_{obs}|\theta_{cand}))$
        \State $i \leftarrow i + 1$
        \EndIf
        \EndIf
        \EndWhile
        \State Return $(\theta_1,\ldots,\theta_{N_{MH}})$, $(w_1,\ldots,w_{N_{MH}})$
        \EndProcedure
    \end{algorithmic}
\end{algorithm}
The only difference to the original Metropolis-Hastings algorithm is that the following algorithm
only accepts parameter values $\theta$ that instantiate DTMC models which satisfy the property of
interest.
\begin{algorithm}[H]
    \caption{Sequential Monte Carlo with rational functions}
    \label{alg:rf-smc}
    \footnotesize{
        \hspace*{\algorithmicindent} \textbf{Input:}
        \begin{itemize}[noitemsep,topsep=0pt]
            \item $\mathcal{M}_\theta$: parametric Discrete-Time Markov chain of parameter $\theta$
            \item $\Phi$: bounded reachability property of interest.
            \item $\pi(\theta)$: prior distribution on $\theta$.
            \item $N$: number of particles in the Sequential Monte Carlo trace.
            \item $M$ perturbation kernels $F_t(\theta^t | \theta^{t-1}_1,\ldots,\theta^{t-1}_N),
                      1\leq t \leq M$
            \item $N_{MH}$: number of particles in each Metropolis-Hastings step.
            \item $Q_t(\theta^t|\theta^{t-1}), 1 \leq t \leq N_{MH}$: transition kernel for
                  Metropolis-Hastings step.
        \end{itemize}
        \hspace*{\algorithmicindent} \textbf{Output:}
        \begin{itemize}[noitemsep,topsep=0pt]
            \item $(\theta_1,\ldots,\theta_N)$, $(w_1,\ldots,w_N)$: $N$ sampled particles and their
                  corresponding weights.
            \item $\hat{\theta}$: estimated model parameter.
        \end{itemize}
    }
    \begin{algorithmic}[1]
        \Procedure{RF-SMC}{}
        \State $i \leftarrow 1$
        \While{$i \leq N$} \algorithmiccomment {SMC initialization}
        \State Draw $\theta$ from $\pi(\theta)$
        \State $\theta_i \leftarrow \theta$
        \State $w_i \leftarrow P(D_{obs}|\theta_i)$
        \State $i \leftarrow i + 1$
        \EndWhile
        \State $t \leftarrow 1$
        \While{$t \leq M$}
        \State $i \leftarrow 1$ \algorithmiccomment{SMC correction step}
        \While{$i \leq N$}
        \State $w'_i \leftarrow \frac{w_i}{\sum_{i=1}^N w_i} $
        \EndWhile
        \State Sample with replacement $(\theta'_1,\ldots,\theta'_N)$ \algorithmiccomment{SMC selection step}
        \NoNumber{\hspace{1.5cm} from $(\theta_1,\ldots,\theta_N)$ with probabilities $(w'_1,\ldots,w'_N)$}
        \State $(\theta_1,\ldots,\theta_N) \leftarrow (\theta'_1,\ldots,\theta'_N)$
        \State $i \leftarrow 1$
        \While{$i \leq N$} \algorithmiccomment {SMC perturbation step}
        \State Draw $\hat{\theta}^t_i$ from $F_t(\theta^t | \theta^{t-1}_1,\ldots,\theta^{t-1}_N), 1\leq t \leq M$
        \State $(\theta^*_1,\ldots,\theta^*_{N_{MH}}), (w^*_1,\ldots,w^*_{N_{MH}}) \leftarrow RF-MH(\hat{\theta}^t_i)$
        \State $\theta_i \leftarrow \theta^*_{N_{MH}}$
        \State $w_i \leftarrow w^*_{N_{MH}}$
        \EndWhile
        \EndWhile
        \State Estimate $\hat{\theta}$ using posterior mean.
        \State Compute $\hat{p}=P(\mathcal{M}_{\hat{\theta}} \models\Phi)$
        \State Return $(\theta_1,\ldots,\theta_{N})$, $(w_1,\ldots,w_{N})$, $\hat{\theta}$, $\hat{p}$
        \EndProcedure
    \end{algorithmic}
\end{algorithm}



\section{SMC-ABC-SMC framework}
Without the availability of analytical form to evaluate the steady-state distribution and the property of
interest, we face the following obstacles:
\begin{itemize}
    \item \textbf{Absence of likelihood functions:} As the rational functions for properties are not
          available, we do not have the analytical form of likelihood. The absence of likelihood
          suggests exploiting \textit{likelihood-free methods}. In this framework, we use Approximate
          Bayesian Computation in combination with the Sequential Monte Carlo method.
    \item \textbf{Absence of rational functions for evaluation of property of interest:} Statistical
          Model Checking then checks the satisfaction of an instantiated model to a bounded path property.
\end{itemize}
For this case, we present Statistical Model Checking, Approximate Bayesian Computation - Sequential
Monte Carlo method \textit{(SMC-ABC-SMC)} framework. SMC-ABC-SMC differs from RF-SMC only in their
perturbation step.
\begin{itemize}
    \item In the SMC-ABC-SMC framework, we work with a \textit{likelihood-free} setup in which there is
          no analytical form to evaluate the likelihood. As there is no likelihood function, we
          apply Approximate Bayesian Computation and  accept the first particle whose simulation runs
          satisfies the distance threshold.
    \item There is also no rational function for the property of interest $\Phi$, so we apply
          Statistical Model Checking with confidence level $\alpha$ and approximation parameter $\delta$.
\end{itemize}

\begin{algorithm}[H]
    \caption{Sequential Monte Carlo with Approximate Bayesian Computation and Statiscal Model Checking}
    \label{smc-abc-smc-alg}
    \footnotesize{
        \hspace*{\algorithmicindent} \textbf{Input:}
        \begin{itemize}[noitemsep,topsep=0pt]
            \item $\mathcal{M}_\theta$: parametric DTMC of parameter $\theta$
            \item $\Phi$: bounded reachability property of interest.
            \item $D_{obs}$: observed data
            \item $\pi(\theta)$: prior distribution on $\theta$.
            \item $N$: number of particles in the Sequential Monte Carlo trace.
            \item $M$ perturbation kernels $F_t(\theta^t | \theta^{t-1}_1,\ldots,\theta^{t-1}_N), 1\leq t \leq M$
            \item $\epsilon$: distance threshold for Approximate Bayesian Computation.
            \item $\delta, \alpha$: indifference and $\alpha$-level for Statistical Model Checking using SPRT method.
        \end{itemize}
        \hspace*{\algorithmicindent} \textbf{Output:}
        \begin{itemize}[noitemsep,topsep=0pt]
            \item $(\theta_1,\ldots,\theta_N), (w_1,\ldots,w_N)$: $N$ sampled particles and their corresponding weights.
            \item $\hat{\theta}$: estimated model parameter.
        \end{itemize}
    }
    \begin{algorithmic}[1]
        \Procedure{SMC-ABC-SMC}{}
        \State $i \leftarrow 1$
        \While{$i \leq N$} \algorithmiccomment {SMC initialization}
        \State Draw $\theta$ from $\pi(\theta)$
        \State $\theta_i \leftarrow \theta$, $w_i \leftarrow 1$
        \EndWhile
        \State $t \leftarrow 1$
        \While{$t \leq M$}
        \State $i \leftarrow 1$ \algorithmiccomment{SMC correction step}
        \While{$i \leq N$}
        \State $w'_i \leftarrow \frac{w_i}{\sum_{i=1}^N w_i} $
        \EndWhile
        \State Sample (with replacement) $(\theta^t_1,\ldots,\theta^t_N)$ \\
        \hspace{3cm} from $(\theta^{t-1}_1,\ldots,\theta^{t-1}_N)$, $(w^{t-1}_1,\ldots,w^{t-1}_N)$  \algorithmiccomment{SMC selection step}
        \State $i \leftarrow 1$
        \While{$i \leq N$} \algorithmiccomment {SMC perturbation step}
        \State $rejected \leftarrow True$
        \While{$rejected == True$}
        \State $sat \leftarrow False $
        \While{$sat = False$}
        \State Draw $\hat{\theta}^t_i$ from $F_t(\theta^t | \theta^{t-1}_1,\ldots,\theta^{t-1}_N), 1\leq t \leq M$
        \If{$SPRT-SMC(\mathcal{M}_{\hat{\theta}^t}, \Phi, \epsilon, \delta)$ is SAT}
        \State $sat \leftarrow True$
        \EndIf
        \EndWhile
        \State Simulate $D_{sim}$ from $(\mathcal{M}_{\hat{\theta}^t})$
        \If{$Distance(D_{sim}, D_{obs}) < \epsilon$}
        \State $rejected \leftarrow False$
        \State $\theta_i \leftarrow \hat{\theta}^t$, $w_i \leftarrow d$
        \EndIf
        \EndWhile
        \EndWhile
        \EndWhile
        \State Estimate $\hat{\theta}$ using posterior mean.
        \State Compute $\hat{p}=P(\mathcal{M}_{\hat{\theta}}\models\Phi)$
        \State Return $(\theta_1,\ldots,\theta_{N})$, $(w_1,\ldots,w_{N})$, $\hat{\theta}$, $\hat{p}$
        \EndProcedure
    \end{algorithmic}
\end{algorithm}

\section{Summary}
In this chapter we presents two Sequential Monte Carlo based frameworks for Bayesian parameter
synthesis of parametric DTMC:
\begin{enumerate}
    \item \textit{RF-SMC}: rational functions based Sequential Monte Carlo, and
    \item \textit{SMC-ABC-SMC}: simulation based Sequential Monte Carlo
\end{enumerate}
In the following chapter, we benchmark the frameworks using different parametric DTMCs to evaluate
the performances of presented frameworks.