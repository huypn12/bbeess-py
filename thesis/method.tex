\chapter{Bayesian frameworks for parameter synthesis.}
\section{Sequential Monte-Carlo without rational functions}
\begin{algorithm}[H]
    \caption{Sequential Monte-Carlo with rational functions}
    \label{smcrf-alg}
    \hspace*{\algorithmicindent} \textbf{Input:}
    \begin{itemize}
        \item $\pi(\theta)$: prior distribution on $\theta$.
        \item $N$: number of particles.
        \item $M$: number of pertubation functions.
        \item $N_{MH}$: number of particles for Metropolis-Hastings in each pertubation.
        \item $D_{obs}$: observed data for Bayesian inference or its summary statistic $S_{obs}$
    \end{itemize}
    \hspace*{\algorithmicindent} \textbf{Output:}
    \begin{itemize}
        \item $(\theta_1,\ldots,\theta_N)$: $N$ sampled particles.
        \item $(w_1,\ldots,w_N)$: corresponding weights of sampled particles.
    \end{itemize}
    \begin{algorithmic}[1]
        \Procedure{Approximate-Bayesian-Computation}{$D$, $\theta$, $\pi(\theta)$, $N$, $\epsilon$}
        \State $t:=0$
        \While{$t \leq N$}
        \EndWhile
        \EndProcedure
    \end{algorithmic}
\end{algorithm}


\section{Sequential Monte-Carlo without rational functions}
Without the availability of analytical form of observational and interested properties, we face the
following obstacles:
\begin{itemize}
    \item \textbf{Absence of likelihood functions:} As the rational functions for properties are not
    available, we do not have the analytical form of likelihood. The abscence of likelihood suggests
    to exploit \textit{likelihood-free methods}. In this framework we use \textit{Approximate
    Bayesian Computation} in combination with \textit{Sequential Monte-Carlo method}.
    \item \textbf{Absence of rational function for verification of bounded path property:} the
    satisfaction of an instantiated model to a bounded path property cannot be computed. In the case
    that the number of states is too large, we use \textit{Statistical Model Checking}.
\end{itemize}


\begin{algorithm}[H]
    \caption{Sequential Monte-Carlo with Approximate Bayesian Computation and Statiscal Model Checking}
    \label{smcabcsmc-alg}
    \hspace*{\algorithmicindent} \textbf{Input:}
    \begin{itemize}
        \item $\pi(\theta)$: prior distribution on $\theta$.
        \item $N$: number of particles.
        \item $M$: number of pertubation functions.
        \item $N_{MH}$: number of particles for Metropolis-Hastings in each pertubation.
        \item $D_{obs}$: observed data for Bayesian inference or its summary statistic $S_{obs}$
        \item $\epsilon$: distance threshold for Approximate Bayesian Computation step.
        \item $\alpha$: confidence interval for Statistical Model Checking step.
    \end{itemize}
    \hspace*{\algorithmicindent} \textbf{Output:}
    \begin{itemize}
        \item $(\theta_1,\ldots,\theta_N)$: $N$ sampled particles.
        \item $(w_1,\ldots,w_N)$: corresponding weights of sampled particles.
    \end{itemize}
    \begin{algorithmic}[1]
        \Procedure{Approximate-Bayesian-Computation}{$D$, $\theta$, $\pi(\theta)$, $N$, $\epsilon$}
        \State $t:=0$
        \While{$t \leq N$}
        \EndWhile
        \EndProcedure
    \end{algorithmic}
\end{algorithm}

\section{Selection of pertubation kernel}
Selection of pertubation kernel is mentioned in \cite{filippi2013optimality}. In this thesis, we use
component-wise uniform kernel:
