\chapter{Introduction}
In different areas of research and application, the objects are to study how the number of
individuals in a closed environment develop under a certain set of assumptions. For instance
\begin{itemize}
      \item Number of online nodes in a distributed system.
      \item Number of surviving individuals in an epidemic model.
\end{itemize}
Markov population models \cite{kingman1969markov} are finite state-space, stochastic models that is
widely used in modeling complex and dynamic systems. In a Markov population model, each state
represents the number of individuals. Formally, in a Markov population model whose state space is
$S=(s_1,\ldots,s_n)$, there is a map $f:S\rightarrow\{0,\ldots,N\}$ where $N\in\mathbf{N}^*$ is the
maximum number of individuals in the system.\\
In a Markov population models, for example Discrete-time Markov Chain, initial and transition
probabilities are known a-priori. In order to encompass unknown attributes of a system, we introduce
\textit{parametric Markov population models}. In a parametric Markov population model, each
transition is a rational function of parameters. As unknown features of the system are represented
by parameters, the following research questions are raised
\begin{itemize}
      \item Given a set of data collected by observing the system, how can we know about its
            parameters?
      \item  Which values of parameters instantiate a model that satisfies a certain property of
            interest?
\end{itemize}
Parameter synthesis is an emerging research direction on probabilistic model checking. Katoen
\cite{katoen2016probabilistic} define the parameter synthesis problem for pDTMC as to find a set of
parameter values, which satisfy a given reachability property. In this thesis, we combines Bayesian
parameter inference and parameter synthesis, so that the result parameters (1) satisfy the property
of interest, and (2) likely to produce given steady-state data. Contributions of the thesis are
\begin{itemize}
      \item Presenting and implementing a data-informed, Bayesian frameworks on parameter synthesis
            of parametric Discrete-time Markov Chain with different case studies.
      \item Comparing the performances of optimization methods used to approximate posterior
            distribution in both cases: closed-form solutions are available and only simulations are
            possible.
      \item Evaluating the scalability of the frameworks with different sizes of model state-space.
\end{itemize}

\begin{itemize}
      \item \textbf{Chapter 1} introduces motivations and goals of this research.
      \item \textbf{Chapter 2} presents the theoretical background on probabilistic model checking,
            include discrete stochastic models and their  corresponding temporal logics.
      \item \textbf{Chapter 3} presents essential concepts on Bayesian inference, including sampling
            and optimization algorithms.
      \item \textbf{Chapter 4} reviews the state-of-the-art works of other researchers on the
            problem of parameter synthesis.
      \item \textbf{Chapter 5} present Bayesian parameter synthesis frameworks.
      \item \textbf{Chapter 6} describes case studies and benchmarks presented frameworks under
            different setups.
      \item \textbf{Chapter 7} conclusion and possible future works.
\end{itemize}

