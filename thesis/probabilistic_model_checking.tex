\chapter{Probabilistic model checking}
 {\color{red}
  \begin{itemize}
      \item Discrete time Markov chain
      \item Continuous time Markov chain, conversion to discrete time chain
      \item Probabilistic Temporal logics
      \item Probabilistic model checking
      \item Statistical model checking
      \item Parametric Discre time Markov chain
      \item Parameter synthesis problem: approaches: bayesian, hhgfhghhh Abate paper on bayesian hhhhhhhhhhhhhhhhhhhhhhhhhhhhhhhhhhhhjjjhhhhhhh
  \end{itemize}
 }

In this thesis, we model stochastic systems. Thus, we use probabilistic models, in our case we mostly use Discrete time markov chain, and also have a

\section{Markov chain}
\subsection{Discrete Time Markov chain}
Our definition of markov chain follows the definition on \cite{baier2008principles}.
\begin{definition}[Discrete Time Markov Chain]
    A Discrete-time Markov chain (DTMC) is a tuple $(S,\mathbf{P}, s_{init}, AP, L)$ where
    \begin{itemize}
        \item $S$ is a countable, non-emty set of \textit{states}
        \item $\mathbf{P}:S\times S \rightarrow [0,1]$ is the \textit{transition probability}
              function such that
              \begin{align*}
                  \forall s \in S : \sum_{s'\in S}\mathbf{P}(s, s') = 1
              \end{align*}
        \item $s_{init}: S \rightarrow [0,1]$ is the \textit{initial distribution} such that
              \begin{align*}
                  \sum_{s\in S}s_{init}(s) = 1
              \end{align*}
        \item $AP$ is a set of \textit{atomic propositions}
        \item $L: S \rightarrow 2^{AP}$ is the labelling function on states.
    \end{itemize}
\end{definition}


\subsection{Continuous-time Markov chain}
Continous-time Markov chain also satisfies memoryless property
\begin{definition}[Continuous-time Markov property]
    Let X be a continuous random variable of exponentially distribution. X has memoryless property if and only if
    \begin{align*}
        Pr\{X > t + \delta | X > t\} = Pr\{X > \delta\} \forall t,\delta\in\mathbb{R}_{\geq 0}
    \end{align*}
\end{definition}

The following definition of Contiuous-time Markov chain is based on \cite{baier2003model}
\begin{definition}[Continuous-time Markov chain]
    A Continuous-time Markov chain (CTMC) is a tuple $(S,\mathbf{P}, \mathbf{r}, S_{init}, AP, L)$ \cite{baier2003model}
    \begin{itemize}
        \item $S$ is a countable, non-emty set of \textit{states}
        \item $\mathbf{P}:S\times S \rightarrow [0,1]$ is the \textit{transition probability}
              function such that
              \begin{align*}
                  \forall s \in S : \sum_{s'\in S}\mathbf{P}(s, s') = 1
              \end{align*}
        \item $\mathbf{r}:S \rightarrow \mathbb{N}$ is the \textit{transition probability}
              function such that
              \begin{align*}
                  \forall s \in S : \sum_{s'\in S}\mathbf{P}(s, s') = 1
              \end{align*}
        \item $s_{init}: S \rightarrow [0,1]$ is the \textit{initial distribution} such that
              \begin{align*}
                  \sum_{s\in S}s_{init}(s) = 1
              \end{align*}
        \item $AP$ is a set of \textit{atomic propositions}
        \item $L: S \rightarrow 2^{AP}$ is the labelling function on states.
    \end{itemize}
\end{definition}


\section{Probabilistic temporal logic}
%% DEFINE CTL property
Over CTL properties, we define the set of PCTL properties, in which we ask the probability to have a CTL property satisfied.
%% DEFINE PCTL property
\begin{definition}[PCTL syntax] The syntax of PCTL is defined as follow
    \begin{align*}
        \Phi & ::== \text{true} \;|\; a \;|\; \Phi \;|\; \Phi \wedge \Phi \;|\; \Phi \vee \Phi \;|\;  P_{\sim  p}[\phi] \\
        \phi & ::== X\Phi \;|\; \Phi U \Phi
    \end{align*}
\end{definition}


\section{Parametric model}
We introduce parameters to formalize unknown attributes of the system.
\begin{definition}[Polynomial ring]
    Given a tuple $\mathbf{x}=(x_1,\ldots,x_n)$ be a tuple
\end{definition}

\begin{definition}{Rational functions}
    Let $\mathbf{x}=\{x_1,\ldots,x_n\}$ be a variable.\\
    Let $\mathbf{Pol}[\mathbf{x}]$ be the set of all polynomial functions over $\mathbf{x}$.\\
    Given $f,g\in\mathbf{Pol}[\mathbf{x}]$, then $h:=\frac{f(\mathbf{x})}{g(\mathbf{x})}, g{\mathbf{x}}\neq 0$ is a rational function over $\mathbf{x}$.\\
    We denote $\mathbb{Q}(\mathbf{x})$ the set of rational functions over $\mathbf{x}$.
\end{definition}


\subsection{Parametric Discrete Time Markov chain}
With the set of rational functions formally defined, we define parametric Discrete-time Markov chain based the definition on \cite{junges2019parameter}.
\begin{definition}[Discrete Time Markov Chain]
    A Discrete-time Markov chain (DTMC) is a tuple $(S, \mathbf{x}, \mathbf{P}, s_{init}, AP, L)$ where
    \begin{itemize}
        \item $S$ is a countable, non-emty set of \textit{states}
        \item $\mathbf{x} \in \mathbb{R}^n, n \in \mathbb{N}$ as the set of $n$ real parameters.
        \item $\mathbf{P}:S\times S \rightarrow \mathbb{Q}(\mathbf{x})$ is the \textit{transition probability}
              function such that
              \begin{align*}
                  \forall s \in S : \sum_{s'\in S}\mathbf{P}(s, s') = 1
              \end{align*}
        \item $s_{init}: S \rightarrow [0,1]$ is the \textit{initial distribution} such that
              \begin{align*}
                  \sum_{s\in S}s_{init}(s) = 1
              \end{align*}
        \item $AP$ is a set of \textit{atomic propositions}
        \item $L: S \rightarrow 2^{AP}$ is the labelling function on states.
    \end{itemize}
\end{definition}

Given a parametric Discrete-time Markov chain $M_p$. A concrete assignment of parameter $\mathbf{x}$ \textit{instantiate} a non-parametric Discrete-time Markov chain if $f{\mathbf{x}}$ evaluates to a real value for all $f\in\mathbf{P}$.