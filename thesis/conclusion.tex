\chapter{Conclusion}
We presented frameworks to perform data-informed parameter synthesis. The frameworks are tested
against different case studies and show that they are able to deliver both satisfying parameter
value set and an estimated parameter value that is close to the original value which is used to
synthesize test data. Therefore, these frameworks are applicable when we need not only an estimation
on the unknown attributes of a system, but also proactively verify the system against a desired property.\\
There are many possible extensions to the presented frameworks, including but not limited to:
\begin{itemize}
      \item \textit{Statistical Model Checking}: we can use Absolute-Error Massart Bounds (proposed by
            Molyneux \cite{molyneux2020abc}, but currently not supported by PRISM) on Statistical
            Model Checking to achieve a better bound on the number of simulation.
      \item \textit{Bayesian Model Checking}: Jha \cite{jha2009bayesian} and Zuliani
            \cite{zuliani2013bayesian} presents a novel method that improves Statistical Model
            Checking by using Bayesian approach.
      \item \textit{Sampling algorithms}: different sampling algorithms can be used to estimate
            posterior distribution. For example, PyMC3 library \cite{salvatier2016pymc3} uses
            No-U-Turn Sampling algorithm by default to estimate posterior distribution.
      \item \textit{Implementation improvement}: currently StormPy prohibits our implementation to be
            parallelized, since StormPy's core classes are not serializable. Porting to C++ language
            possibly has several benefits by achieving higher performance of C++ and exploiting the
            data-parallelism of Sequential Monte Carlo algorithm.
\end{itemize}