\chapter{Introduction}
\section{Motivation}
Firstly introduced by Kingsman \cite{kingman1969markov}, Markov population processes are finite
state-space, stochastic models widely used in modeling complex and dynamical systems. In a Markov
population model, each state represents the number of individuals, and the transitions among states
represent the increase or decrease of a population. In general, Markov population models study the
population dynamics of a system of interest. For example, Markov population processes are able to
model:
\begin{itemize}
      \item Number of online nodes in a distributed system.
      \item Number of surviving individuals in an epidemic model.
\end{itemize}
Studying the Markov population model has challenges. First, model state space exponentially expands
as we capture more attributes and behavior of the system of interest. The explosion of state-space
makes model checking of the Markov population model computationally intensive. Second, in a Markov
population model, such as Discrete-time Markov Chain, initial and transition probabilities are known
a priori. To encompass unknown attributes of a system, we introduce parametric Markov population
models. In a parametric Markov population model, each transition is a rational function of
parameters. As parameters represent unknown features of the system, it gives the following research
questions
\begin{itemize}
      \item Given a set of data collected by observing the system, what can we know about its
            parameters?
      \item Which values of parameters instantiate a model that satisfies a specific property of
            interest?
\end{itemize}
Parameter synthesis is an emerging research direction on probabilistic model checking. Katoen
\cite{katoen2016probabilistic} defines the parameter synthesis problem for the parametric
discrete-time Markov chain to find a set of parameter values that satisfy a given reachability
property. In this thesis, we combine Bayesian parameter inference and parameter synthesis. The
result parameters (i) satisfy the property of interest, and (ii) are likely to produce given
steady-state data. Contributions of the thesis are
\begin{itemize}
      \item We are presenting and implementing a data-informed, Bayesian framework on parameter
            synthesis of parametric Discrete-time Markov chain. The frameworks work in two cases:
            (i) when the exact likelihood function of the property of interest is available, and
            (ii) when it has to be approximated utilizing Monte-Carlo methods.
      \item We compare the performances of proposed frameworks used to approximate posterior
            distribution on various case studies.
      \item We evaluate the scalability of the frameworks for different sizes of model state-space.
\end{itemize}

\section{Related works}
%% Frameworks from molyneux
The frameworks presented in this thesis are based on ABC-SMC framework \cite{molyneux2019bayesian}
and ABC-(SMC)2 \cite{molyneux2020abc} by Molyneux et al. However, the ABC-SMC and ABC-(SMC)2
frameworks synthesize parameters for CTMC and check the CTMC model against CSL properties. In
parametric DTMC, since the symbolic rational function of PCTL property is obtainable
\cite{daws2004symbolic}, we based on Del Moral \cite{del2006sequential} and Daviet
\cite{daviet2018inference} to construct an algorithm based on evaluation of symbolic rational
function, then benchmark it against a simulation-based approach.

%% Model checking
The theoretical background of model checking discrete-time Markov chain is presented by Baier et al.
\cite{baier2008principles}. Katoen \cite{katoen2016probabilistic} presents a tutorial to model check
parametric discrete-time Markov chain and current methods on parameter synthesis. More in-depth
surveys and discoveries on parametric model checking and parameter synthesis are presented by Junges
\cite{junges2020parameter} and Hutschenreiter \cite{hutschenreiter2017parametric}.

%% Bayesian inference: optimzation methods
Markov Chain Monte Carlo sampling algorithms used in this thesis are presented by Metropolis
\cite{metropolis1953equation}, and Hastings \cite{hastings1970monte}. Del Moral
\cite{del2006sequential} designed Sequential Monte Carlo to address the limitations of Markov Chain
Monte Carlo. A comparison between different Monte Carlo sampling algorithms, including Markov-chain
Monte Carlo and Sequential Monte Carlo is presented in \cite{daviet2018inference}. Silk
\cite{silk2012optimizing} and Filippi \cite{filippi2013optimality} discussed different approaches on
the perturbation kernel selection of Sequential Monte Carlo and Sequential Monte Carlo with
Approximate Bayesian Computation algorithms.

%% Tools: PRISM and STORM
The model checking step in the frameworks presented by this thesis is implemented using Storm model
checker \cite{hensel2020probabilistic}. Storm provides well-documented and easy-to-use APIs to embed
model checking to software projects programmatically. However, Storm does not support Statistical
Model Checking. Thus, the Statistical Model Checking step in simulation-based frameworks is
implemented using PRISM \cite{kwiatkowska2011prism}.

\section{Structure of the thesis}
The content in this thesis is organized to 7 chapters:
\begin{itemize}
      \item \textbf{Chapter 1} introduces motivations and goals of this research.
      \item \textbf{Chapter 2} presents the theoretical background on probabilistic model checking,
            include discrete stochastic models and their  corresponding temporal logics.
      \item \textbf{Chapter 3} presents essential concepts on Bayesian inference, including sampling
            and optimization algorithms.
      \item \textbf{Chapter 4} proposes Bayesian parameter synthesis frameworks.
      \item \textbf{Chapter 5} describes case studies and benchmarks presented frameworks under
            different setups.
      \item \textbf{Chapter 6} conclusion and outlook.
\end{itemize}

